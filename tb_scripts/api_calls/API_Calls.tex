% Options for packages loaded elsewhere
\PassOptionsToPackage{unicode}{hyperref}
\PassOptionsToPackage{hyphens}{url}
\PassOptionsToPackage{dvipsnames,svgnames,x11names}{xcolor}
%
\documentclass[
  letterpaper,
  DIV=11,
  numbers=noendperiod]{scrartcl}

\usepackage{amsmath,amssymb}
\usepackage{lmodern}
\usepackage{iftex}
\ifPDFTeX
  \usepackage[T1]{fontenc}
  \usepackage[utf8]{inputenc}
  \usepackage{textcomp} % provide euro and other symbols
\else % if luatex or xetex
  \usepackage{unicode-math}
  \defaultfontfeatures{Scale=MatchLowercase}
  \defaultfontfeatures[\rmfamily]{Ligatures=TeX,Scale=1}
\fi
% Use upquote if available, for straight quotes in verbatim environments
\IfFileExists{upquote.sty}{\usepackage{upquote}}{}
\IfFileExists{microtype.sty}{% use microtype if available
  \usepackage[]{microtype}
  \UseMicrotypeSet[protrusion]{basicmath} % disable protrusion for tt fonts
}{}
\makeatletter
\@ifundefined{KOMAClassName}{% if non-KOMA class
  \IfFileExists{parskip.sty}{%
    \usepackage{parskip}
  }{% else
    \setlength{\parindent}{0pt}
    \setlength{\parskip}{6pt plus 2pt minus 1pt}}
}{% if KOMA class
  \KOMAoptions{parskip=half}}
\makeatother
\usepackage{xcolor}
\setlength{\emergencystretch}{3em} % prevent overfull lines
\setcounter{secnumdepth}{-\maxdimen} % remove section numbering
% Make \paragraph and \subparagraph free-standing
\ifx\paragraph\undefined\else
  \let\oldparagraph\paragraph
  \renewcommand{\paragraph}[1]{\oldparagraph{#1}\mbox{}}
\fi
\ifx\subparagraph\undefined\else
  \let\oldsubparagraph\subparagraph
  \renewcommand{\subparagraph}[1]{\oldsubparagraph{#1}\mbox{}}
\fi

\usepackage{color}
\usepackage{fancyvrb}
\newcommand{\VerbBar}{|}
\newcommand{\VERB}{\Verb[commandchars=\\\{\}]}
\DefineVerbatimEnvironment{Highlighting}{Verbatim}{commandchars=\\\{\}}
% Add ',fontsize=\small' for more characters per line
\usepackage{framed}
\definecolor{shadecolor}{RGB}{241,243,245}
\newenvironment{Shaded}{\begin{snugshade}}{\end{snugshade}}
\newcommand{\AlertTok}[1]{\textcolor[rgb]{0.68,0.00,0.00}{#1}}
\newcommand{\AnnotationTok}[1]{\textcolor[rgb]{0.37,0.37,0.37}{#1}}
\newcommand{\AttributeTok}[1]{\textcolor[rgb]{0.40,0.45,0.13}{#1}}
\newcommand{\BaseNTok}[1]{\textcolor[rgb]{0.68,0.00,0.00}{#1}}
\newcommand{\BuiltInTok}[1]{\textcolor[rgb]{0.00,0.23,0.31}{#1}}
\newcommand{\CharTok}[1]{\textcolor[rgb]{0.13,0.47,0.30}{#1}}
\newcommand{\CommentTok}[1]{\textcolor[rgb]{0.37,0.37,0.37}{#1}}
\newcommand{\CommentVarTok}[1]{\textcolor[rgb]{0.37,0.37,0.37}{\textit{#1}}}
\newcommand{\ConstantTok}[1]{\textcolor[rgb]{0.56,0.35,0.01}{#1}}
\newcommand{\ControlFlowTok}[1]{\textcolor[rgb]{0.00,0.23,0.31}{#1}}
\newcommand{\DataTypeTok}[1]{\textcolor[rgb]{0.68,0.00,0.00}{#1}}
\newcommand{\DecValTok}[1]{\textcolor[rgb]{0.68,0.00,0.00}{#1}}
\newcommand{\DocumentationTok}[1]{\textcolor[rgb]{0.37,0.37,0.37}{\textit{#1}}}
\newcommand{\ErrorTok}[1]{\textcolor[rgb]{0.68,0.00,0.00}{#1}}
\newcommand{\ExtensionTok}[1]{\textcolor[rgb]{0.00,0.23,0.31}{#1}}
\newcommand{\FloatTok}[1]{\textcolor[rgb]{0.68,0.00,0.00}{#1}}
\newcommand{\FunctionTok}[1]{\textcolor[rgb]{0.28,0.35,0.67}{#1}}
\newcommand{\ImportTok}[1]{\textcolor[rgb]{0.00,0.46,0.62}{#1}}
\newcommand{\InformationTok}[1]{\textcolor[rgb]{0.37,0.37,0.37}{#1}}
\newcommand{\KeywordTok}[1]{\textcolor[rgb]{0.00,0.23,0.31}{#1}}
\newcommand{\NormalTok}[1]{\textcolor[rgb]{0.00,0.23,0.31}{#1}}
\newcommand{\OperatorTok}[1]{\textcolor[rgb]{0.37,0.37,0.37}{#1}}
\newcommand{\OtherTok}[1]{\textcolor[rgb]{0.00,0.23,0.31}{#1}}
\newcommand{\PreprocessorTok}[1]{\textcolor[rgb]{0.68,0.00,0.00}{#1}}
\newcommand{\RegionMarkerTok}[1]{\textcolor[rgb]{0.00,0.23,0.31}{#1}}
\newcommand{\SpecialCharTok}[1]{\textcolor[rgb]{0.37,0.37,0.37}{#1}}
\newcommand{\SpecialStringTok}[1]{\textcolor[rgb]{0.13,0.47,0.30}{#1}}
\newcommand{\StringTok}[1]{\textcolor[rgb]{0.13,0.47,0.30}{#1}}
\newcommand{\VariableTok}[1]{\textcolor[rgb]{0.07,0.07,0.07}{#1}}
\newcommand{\VerbatimStringTok}[1]{\textcolor[rgb]{0.13,0.47,0.30}{#1}}
\newcommand{\WarningTok}[1]{\textcolor[rgb]{0.37,0.37,0.37}{\textit{#1}}}

\providecommand{\tightlist}{%
  \setlength{\itemsep}{0pt}\setlength{\parskip}{0pt}}\usepackage{longtable,booktabs,array}
\usepackage{calc} % for calculating minipage widths
% Correct order of tables after \paragraph or \subparagraph
\usepackage{etoolbox}
\makeatletter
\patchcmd\longtable{\par}{\if@noskipsec\mbox{}\fi\par}{}{}
\makeatother
% Allow footnotes in longtable head/foot
\IfFileExists{footnotehyper.sty}{\usepackage{footnotehyper}}{\usepackage{footnote}}
\makesavenoteenv{longtable}
\usepackage{graphicx}
\makeatletter
\def\maxwidth{\ifdim\Gin@nat@width>\linewidth\linewidth\else\Gin@nat@width\fi}
\def\maxheight{\ifdim\Gin@nat@height>\textheight\textheight\else\Gin@nat@height\fi}
\makeatother
% Scale images if necessary, so that they will not overflow the page
% margins by default, and it is still possible to overwrite the defaults
% using explicit options in \includegraphics[width, height, ...]{}
\setkeys{Gin}{width=\maxwidth,height=\maxheight,keepaspectratio}
% Set default figure placement to htbp
\makeatletter
\def\fps@figure{htbp}
\makeatother

\KOMAoption{captions}{tableheading}
\makeatletter
\makeatother
\makeatletter
\makeatother
\makeatletter
\@ifpackageloaded{caption}{}{\usepackage{caption}}
\AtBeginDocument{%
\ifdefined\contentsname
  \renewcommand*\contentsname{Table of contents}
\else
  \newcommand\contentsname{Table of contents}
\fi
\ifdefined\listfigurename
  \renewcommand*\listfigurename{List of Figures}
\else
  \newcommand\listfigurename{List of Figures}
\fi
\ifdefined\listtablename
  \renewcommand*\listtablename{List of Tables}
\else
  \newcommand\listtablename{List of Tables}
\fi
\ifdefined\figurename
  \renewcommand*\figurename{Figure}
\else
  \newcommand\figurename{Figure}
\fi
\ifdefined\tablename
  \renewcommand*\tablename{Table}
\else
  \newcommand\tablename{Table}
\fi
}
\@ifpackageloaded{float}{}{\usepackage{float}}
\floatstyle{ruled}
\@ifundefined{c@chapter}{\newfloat{codelisting}{h}{lop}}{\newfloat{codelisting}{h}{lop}[chapter]}
\floatname{codelisting}{Listing}
\newcommand*\listoflistings{\listof{codelisting}{List of Listings}}
\makeatother
\makeatletter
\@ifpackageloaded{caption}{}{\usepackage{caption}}
\@ifpackageloaded{subcaption}{}{\usepackage{subcaption}}
\makeatother
\makeatletter
\@ifpackageloaded{tcolorbox}{}{\usepackage[many]{tcolorbox}}
\makeatother
\makeatletter
\@ifundefined{shadecolor}{\definecolor{shadecolor}{rgb}{.97, .97, .97}}
\makeatother
\makeatletter
\makeatother
\ifLuaTeX
  \usepackage{selnolig}  % disable illegal ligatures
\fi
\IfFileExists{bookmark.sty}{\usepackage{bookmark}}{\usepackage{hyperref}}
\IfFileExists{xurl.sty}{\usepackage{xurl}}{} % add URL line breaks if available
\urlstyle{same} % disable monospaced font for URLs
\hypersetup{
  pdftitle={API Calls in R},
  pdfauthor={Trevor French},
  colorlinks=true,
  linkcolor={blue},
  filecolor={Maroon},
  citecolor={Blue},
  urlcolor={Blue},
  pdfcreator={LaTeX via pandoc}}

\title{API Calls in R}
\author{Trevor French}
\date{}

\begin{document}
\maketitle
\ifdefined\Shaded\renewenvironment{Shaded}{\begin{tcolorbox}[breakable, interior hidden, sharp corners, frame hidden, borderline west={3pt}{0pt}{shadecolor}, enhanced, boxrule=0pt]}{\end{tcolorbox}}\fi

\hypertarget{install-packages}{%
\subsection{Install Packages}\label{install-packages}}

\begin{Shaded}
\begin{Highlighting}[]
\FunctionTok{install.packages}\NormalTok{(}\FunctionTok{c}\NormalTok{(}\StringTok{\textquotesingle{}httr\textquotesingle{}}\NormalTok{, }\StringTok{\textquotesingle{}jsonlite\textquotesingle{}}\NormalTok{))}
\end{Highlighting}
\end{Shaded}

\hypertarget{require-packages}{%
\subsection{Require Packages}\label{require-packages}}

\begin{verbatim}
Warning: package 'httr' was built under R version 4.1.3
\end{verbatim}

\begin{verbatim}
Warning: package 'jsonlite' was built under R version 4.1.3
\end{verbatim}

\hypertarget{make-request}{%
\subsection{Make Request}\label{make-request}}

Pass a URL into the `GET' function and store the response in a variable
called `res'.

\begin{Shaded}
\begin{Highlighting}[]
\NormalTok{res }\OtherTok{=} \FunctionTok{GET}\NormalTok{(}\StringTok{"https://api.helium.io/v1/stats"}\NormalTok{)}
\FunctionTok{print}\NormalTok{(res)}
\end{Highlighting}
\end{Shaded}

\begin{verbatim}
Response [https://api.helium.io/v1/stats]
  Date: 2022-08-03 03:11
  Status: 200
  Content-Type: application/json; charset=utf-8
  Size: 932 B
\end{verbatim}

\hypertarget{parse-explore-data-manually}{%
\subsection{Parse \& Explore Data
Manually}\label{parse-explore-data-manually}}

Use the `fromJSON' function from the `jsonlite' package to parse the
response data and then print out the names in the resulting data set.

\begin{Shaded}
\begin{Highlighting}[]
\NormalTok{data }\OtherTok{=} \FunctionTok{fromJSON}\NormalTok{(}\FunctionTok{rawToChar}\NormalTok{(res}\SpecialCharTok{$}\NormalTok{content))}

\FunctionTok{names}\NormalTok{(data)}
\end{Highlighting}
\end{Shaded}

\begin{verbatim}
[1] "data"
\end{verbatim}

Go one level deeper into the data set and print out the names again.

\begin{Shaded}
\begin{Highlighting}[]
\NormalTok{data }\OtherTok{=}\NormalTok{ data}\SpecialCharTok{$}\NormalTok{data}

\FunctionTok{names}\NormalTok{(data)}
\end{Highlighting}
\end{Shaded}

\begin{verbatim}
[1] "token_supply"     "election_times"   "counts"           "challenge_counts"
[5] "block_times"     
\end{verbatim}

Alternatively, you can loop through the names as follows.

\begin{Shaded}
\begin{Highlighting}[]
\ControlFlowTok{for}\NormalTok{ (name }\ControlFlowTok{in} \FunctionTok{names}\NormalTok{(data))\{}\FunctionTok{print}\NormalTok{(name)\}}
\end{Highlighting}
\end{Shaded}

\begin{verbatim}
[1] "token_supply"
[1] "election_times"
[1] "counts"
[1] "challenge_counts"
[1] "block_times"
\end{verbatim}

Get the `token\_supply' field from the data.

\begin{Shaded}
\begin{Highlighting}[]
\NormalTok{token\_supply }\OtherTok{=}\NormalTok{ data}\SpecialCharTok{$}\NormalTok{token\_supply}

\FunctionTok{print}\NormalTok{(token\_supply)}
\end{Highlighting}
\end{Shaded}

\begin{verbatim}
[1] 124629266
\end{verbatim}

\hypertarget{adding-parameters-to-requests}{%
\subsection{Adding Parameters to
Requests}\label{adding-parameters-to-requests}}

Add `min\_time' and `max\_time' as parameters on a different endpoint
and print the resulting `fee' data.

\begin{Shaded}
\begin{Highlighting}[]
\NormalTok{res }\OtherTok{=} \FunctionTok{GET}\NormalTok{(}\StringTok{"https://api.helium.io/v1/dc\_burns/sum"}\NormalTok{,}
    \AttributeTok{query =} \FunctionTok{list}\NormalTok{(}\AttributeTok{min\_time =} \StringTok{"2020{-}07{-}27T00:00:00Z"}
\NormalTok{                 , }\AttributeTok{max\_time =} \StringTok{"2021{-}07{-}27T00:00:00Z"}\NormalTok{))}

\NormalTok{data }\OtherTok{=} \FunctionTok{fromJSON}\NormalTok{(}\FunctionTok{rawToChar}\NormalTok{(res}\SpecialCharTok{$}\NormalTok{content))}
\NormalTok{fee }\OtherTok{=}\NormalTok{ data}\SpecialCharTok{$}\NormalTok{data}\SpecialCharTok{$}\NormalTok{fee}
\FunctionTok{print}\NormalTok{(fee)}
\end{Highlighting}
\end{Shaded}

\begin{verbatim}
[1] 10112755000
\end{verbatim}

\hypertarget{adding-headers-to-requests}{%
\subsection{Adding Headers to
Requests}\label{adding-headers-to-requests}}

Execute the same query as above except this time specify headers. This
will likely be necessary when working with an API which requires an API
Key.

\begin{Shaded}
\begin{Highlighting}[]
\NormalTok{res }\OtherTok{=} \FunctionTok{GET}\NormalTok{(}\StringTok{"https://api.helium.io/v1/dc\_burns/sum"}\NormalTok{,}
    \AttributeTok{query =} \FunctionTok{list}\NormalTok{(}\AttributeTok{min\_time =} \StringTok{"2020{-}07{-}27T00:00:00Z"}
\NormalTok{                 , }\AttributeTok{max\_time =} \StringTok{"2021{-}07{-}27T00:00:00Z"}\NormalTok{),}
    \FunctionTok{add\_headers}\NormalTok{(}\StringTok{\textasciigrave{}}\AttributeTok{Accept}\StringTok{\textasciigrave{}}\OtherTok{=}\StringTok{\textquotesingle{}application/json\textquotesingle{}}\NormalTok{, }\StringTok{\textasciigrave{}}\AttributeTok{Connection}\StringTok{\textasciigrave{}}\OtherTok{=}\StringTok{\textquotesingle{}keep{-}live\textquotesingle{}}\NormalTok{))}

\NormalTok{data }\OtherTok{=} \FunctionTok{fromJSON}\NormalTok{(}\FunctionTok{rawToChar}\NormalTok{(res}\SpecialCharTok{$}\NormalTok{content))}
\NormalTok{fee }\OtherTok{=}\NormalTok{ data}\SpecialCharTok{$}\NormalTok{data}\SpecialCharTok{$}\NormalTok{fee}
\FunctionTok{print}\NormalTok{(fee)}
\end{Highlighting}
\end{Shaded}

\begin{verbatim}
[1] 10112755000
\end{verbatim}



\end{document}
